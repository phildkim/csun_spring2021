\documentclass[a4paper]{article}
\usepackage[left=1.0in,top=1.0in,right=1.0in,bottom=1.0in]{geometry}
\usepackage[english]{babel}
\usepackage[utf8]{inputenc}
\usepackage{hyperref,graphicx}
\usepackage{
  amssymb,
  amsmath,
  amsthm,
  latexsym
}
\begin{document}
\section*{Chapter 21}
\begin{itemize}
  \item What is electric charge
  \item Coulomb Force
  \item Superposition of forces
  \item Electric Field
  \begin{itemize}
    \item system of charges
    \item continuous distribution
    \begin{itemize}
      \item linear charge density
      \item surface charge density
      \item volume charge density
      \begin{itemize}
        \item Must know how to derive the expression for electric field for various distributions
      \end{itemize}
    \end{itemize}
  \end{itemize}
  \item particle in uniform electric field
  \item electric dipole moment
  \item torque
\end{itemize}
\newpage
\section*{Chapter 22}
\begin{itemize}
  \item Electric Flux
  \item Gauss's Law
  \begin{itemize}
    \item various gaussian surfaces
    \item finding the electric field for infinite long rod
    \item finding the electric field for charged surface
    \item finding the electric field for charged sphere
    \item finding the electric field for metal sphere
    \item finding the electric field for  insulating sphere
  \end{itemize}
  \item conductors with cavities
\end{itemize}
\newpage
\section*{Chapter 23}
\begin{itemize}
  \item work done by electric force
  \item electric potential energy: total electric potential energy
  \item electric potential energy: work done  found using the change in potential energy
  \item conservation of energy
  \item electric potential (potential)
  \begin{itemize}
    \item potential as a potential energy per unit charge
    \item change potential as a line integral of electric field
    \item voltage as change in potential energy
    \item total potential energy for a system of particles
    \item potential due to continuous distribution
  \end{itemize}
  \item equipotential surfaces
  \item potential gradient
  \begin{itemize}
    \item finding the electric field as a partial derivative of the potential as a function of position
  \end{itemize}
\end{itemize}
\end{document}
