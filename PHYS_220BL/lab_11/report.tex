\documentclass{article}
\usepackage[left=0.5in,top=0.5in,right=0.5in,bottom=0.5in]{geometry}
\usepackage[english]{babel}
\usepackage[utf8]{inputenc}
\usepackage[table]{xcolor}
\usepackage{amssymb,amsmath,amsthm}
\usepackage{changepage,threeparttable}
\usepackage{booktabs,multirow}
\usepackage{graphicx}
\usepackage{soul}
\graphicspath{{./images/}}
\def\F#1{\(#1\)}
\title{Lab 11: Electron Acceleration and Deflection by Electrostatic Fields}
\author{Philip Kim}
\date{\today}
\begin{document}
\maketitle
\vspace*{-1cm}
\begin{table}[!htp]\centering
  \begin{tabular}{|c|c|c|c|}\hline
    \multicolumn{4}{|c|}{\textbf{Table 2: Electron Deflection}}\\\hline
    \F{X_{obs}}&\F{Y_{obs}}&Y&\F{F_D}\\\hline
    8.3&-2.40, 2.35&2.38&-0.03\\\hline
    8.0&-2.20, 2.05&2.13&-0.08\\\hline
    7.3&-2.10, 1.45&1.78&-0.33\\\hline
    7.0&-1.40, 1.35&1.38&-0.03\\\hline
    6.3&-1.30, 1.25&1.28&-0.03\\\hline
    6.0&-1.20, 1.15&1.18&-0.03\\\hline
    5.3&-1.10, 1.10&1.10&0.00\\\hline
    5.0&-0.50, 1.00&0.75&0.25\\\hline
    4.3&-0.40, 0.50&0.45&0.05\\\hline
    4.0&-0.30, 0.30&0.30&0.00\\\hline
  \end{tabular}
\end{table}
\begin{itemize}
  \item[(a)] Measure the distance between the plates s = \F{\boxed{5.3~cm}}
  \item[(b)] Graph \F{y~vs.~x^2}, include error bars. Measure the slope of the graph, slope = \F{\boxed{0.0307}}
  \item[] \includegraphics[scale=0.25]{graph.jpg}
  \item[(c)] From the slope, calculate the correction factor \F{F_D=\boxed{0.023}}
\end{itemize}
\newpage
\begin{table}[!htp]\centering
  \begin{tabular}{|c|c|c|c|c|}\hline
    \multicolumn{5}{|c|}{\textbf{Table 3: Thompson's Experiment}}\\\hline
    \F{V_{PS} (kV)}&2.00&2.50&3.00&3.50\\\hline
    \F{I} (A)&0.18&0.23&0.28&0.33\\\hline
    \F{B} (T)&7.62e-4&9.73e-4&0.12e-2&0.14e-2\\\hline
    \F{e/m} (C/kg)& & & & \\\hline
  \end{tabular}
\end{table}
Sketch the path of the beam:\\
\fbox{\begin{minipage}{53em}
  \includegraphics[width=\textwidth]{beam.jpeg}
\end{minipage}}
\subsection*{11.6 Questions}
\begin{enumerate}
  \item Calculate the speed of the electron for the maximum voltage available for acceleration, in meters per seconds.\\
  \F{v=\sqrt{\frac{2qV}{m}}\rightarrow\sqrt{\frac{2\cdot(1.6\times{10}^{-19}\times{10}^3)\cdot(3.5)}{9.1\times{10}^{-31}}}=\boxed{3.51\times{10}^7~m/s}}
  \item What fraction of the speed of light is this?\\
  \F{\frac{v}{c}=\frac{3.51\times{10}^7}{3\times{10}^{-8}}=\boxed{0.117}}
  \item According to the special theory of relativity, the mass m of an object that is moving with velocity v with respect to an observer is larger than its rest mass \F{m_0}. The rest mass is the mass of the object when it is at rest. The equation that describes this phenomenon is \begin{center}\F{m=\frac{m_0}{\sqrt{1-\frac{v^2}{c^2}}}},\end{center} where \F{c=3.0\times10^8m/s} is the speed of light in vacuum. Evaluate the mass for the electrons in this experiment that are moving at v you calculated in 1. How much larger is this than \F{m_0}?\\
  \F{m=\frac{9.1\times{10}^{-31}}{\sqrt{1-\frac{{(3.51\times{10}^7)}^2}{{(3\times{10}^{-8})}^2}}}=\boxed{9.1629\times{10}^{-31} kg}\qquad \frac{m}{m_0}\rightarrow~\frac{9.1629\times{10}^{-31}}{9.1\times{10}^{-31}}=1.0069\rightarrow~\boxed{0.69\%~larger~than~m_0}}
  \item Compare your measured \F{e/m} from Thompsons Experiment to the known value, \F{1.76\times10^{11}C/kg}.\\
  \F{\frac{e}{m}=\frac{1.6\times{10}^{-19}}{9.1629\times{10}^{-31}}=\boxed{1.747\times{10}^{11}~C/kg}}
\end{enumerate}
\end{document}
