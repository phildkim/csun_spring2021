\documentclass[a4paper]{article}
\usepackage[left=1.0in,top=1.0in,right=1.0in,bottom=1.0in]{geometry}
\usepackage[english]{babel}
\usepackage[utf8]{inputenc}
\usepackage{amssymb,amsmath,amsthm,latexsym}
\begin{document}
\begin{center}
  \section*{Gauss's Law}
\end{center}
\begin{equation}
  \begin{split}
  \Phi_E &= \oint \overrightarrow{E} \cdot d\overrightarrow{A} \\
         &= \frac{q_{encl}}{\epsilon_0},~~where~~\epsilon_0 = 8.85\times 10^{-12}\frac{C^2}{N\cdot m^2}
  \end{split}
\end{equation}
\end{document}
% Single point charge q	Distance r from q	E=14πϵ0qr2
% Charge q on surface of conducting sphere with radius R	Outside sphere, r>R	E=14πϵ0qr2
% Inside sphere, r<R	E=0
% Infinite wire, charge per unit length λ	Distance r from wire	E=12πϵ0λr
% Infinite conducting cylinder with radius R, charge per unit length λ	Outside cylinder, r>R	E=12πϵ0λr
% Inside cylinder, r<R	E=0
% Solid insulating sphere with radius R, charge Q distributed uniformly throughout volume	Outside sphere, r>R	E=14πϵ0Qr2
% Inside sphere, r<R	E=14πϵ0QrR3
% Infinite sheet of charge with uniform charge per unit area σ	Any point	E=σ2ϵ0
% Two oppositely charged conducting plates with surface charge densities +σ and −σ	Any point between plates	E=σϵ0
% Charged conductor	Just outside the conductor	E=σϵ0